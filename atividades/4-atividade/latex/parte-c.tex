
\subsection{Parte (c): Análise de Estabilidade com o Critério de Routh-Hurwitz}

Após calcular a função de transferência em malha fechada \(C(s)/R(s)\), o próximo passo é analisar a estabilidade do sistema de controle. Utilizamos o critério de Routh-Hurwitz para essa finalidade, que é uma técnica fundamental na teoria de controle para determinar a estabilidade de um sistema linear.


\subsubsection{Código Scilab para a Análise de Routh-Hurwitz}
\begin{lstlisting}[language=Scilab, caption=Código Scilab para calcular a matriz de Routh-Hurwitz]
    // Definicao dos parametros
    s = poly(0, 's');
    m = 10;  // massa
    c = 7;   // coeficiente de amortecimento
    k = 5;   // constante da mola

    // Definicao das funcoes de transferencia
    K = m / 3;  // Ganho do controlador proporcional
    Ts = m / 6;  // Constante de tempo do sensor

    // Funcoes de Transferencia
    Gp = syslin('c', 1, m*s^2 + c*s + k);  // Planta
    Hs = syslin('c', 1, Ts*s + 1);  // Sensor
    Gc = syslin('c', K, 1);  // Controlador Proporcional

    // Funcao de Transferencia em Malha Fechada C(s)/R(s)
    sys = Gc * Gp / (1 + Gc * Gp * Hs);

    // Extraindo o denominador da Funcao de Transferencia para Analise de Estabilidade
    den = sys.den;

    rh_matrix = routh_t(den);

    // Exibir a Matriz de Routh-Hurwitz
    disp("Matriz de Routh-Hurwitz:");
    disp(rh_matrix);
    \end{lstlisting}


\subsubsection{Extração do Denominador da Função de Transferência}

A estabilidade do sistema pode ser analisada através do denominador da função de transferência em malha fechada, dado que as raízes do polinômio característico no denominador determinam a resposta do sistema. O denominador extraído é:
\[ 0.62 + 1s + 1.1s^2 + s^3 \]


\subsubsection{Cálculo da Matriz de Routh-Hurwitz}

Para construir a matriz de Routh-Hurwitz, utilizamos a função \texttt{routh\_t} que calcula essa matriz a partir do polinômio característico. A matriz de Routh-Hurwitz para o denominador do sistema é calculada e apresentada como segue:
\[
    \begin{array}{c|cc}
        s^3 & 1          & 0.92 \\
        s^2 & 1.3        & 0.5  \\
        s^1 & 0.5353846  & 0    \\
        s^0 & 0.5        &      \\
    \end{array}
\]

\subsubsection{Interpretação da Matriz de Routh-Hurwitz}

Os elementos da primeira coluna da matriz de Routh-Hurwitz indicam a estabilidade do sistema. Todos os elementos devem ser positivos para garantir estabilidade. A matriz mostra que todos os termos são positivos, sugerindo que o sistema é estável. Esta análise detalhada fornece confiança adicional na robustez do sistema sob a configuração de controle atual.

\subsubsection{Conclusão da Análise de Estabilidade}

A análise com a matriz de Routh-Hurwitz confirma que o sistema é estável sob as condições atuais. A positividade de todos os termos na primeira coluna da matriz assegura que não há raízes com partes reais positivas, o que implica em uma resposta do sistema estável e controlada. Essa conclusão é vital para garantir que o sistema opere de forma segura e eficaz, mantendo o desempenho desejado.