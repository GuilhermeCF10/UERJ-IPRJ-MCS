
\subsection{Parte (c): Análise de Estabilidade com o Critério de Routh-Hurwitz}

Após calcular a função de transferência em malha fechada \(C(s)/R(s)\), o próximo passo é analisar a estabilidade do sistema de controle. Utilizamos o critério de Routh-Hurwitz para essa finalidade, que é uma técnica fundamental na teoria de controle para determinar a estabilidade de um sistema linear.


\subsubsection{Código Scilab para a Análise de Routh-Hurwitz}
\begin{lstlisting}[language=Scilab, caption=Código Scilab para calcular a matriz de Routh-Hurwitz]
    // Definicao dos parametros
    s = poly(0, 's');
    m = 10;  // massa
    c = 7;   // coeficiente de amortecimento
    k = 5;   // constante da mola

    // Definicao das funcoes de transferencia
    K = m / 3;  // Ganho do controlador proporcional
    Ts = m / 6;  // Constante de tempo do sensor

    // Funcoes de Transferencia
    Gp = syslin('c', 1, m*s^2 + c*s + k);  // Planta
    Hs = syslin('c', 1, Ts*s + 1);  // Sensor
    Gc = syslin('c', K, 1);  // Controlador Proporcional

    // Funcao de Transferencia em Malha Fechada C(s)/R(s)
    sys = Gc * Gp / (1 + Gc * Gp * Hs);

    // Extraindo o denominador da Funcao de Transferencia para Analise de Estabilidade
    den = sys.den;

    rh_matrix = routh_t(den);

    // Exibir a Matriz de Routh-Hurwitz
    disp("Matriz de Routh-Hurwitz:");
    disp(rh_matrix);
\end{lstlisting}


\subsubsection{Extração do Denominador da Função de Transferência}

A estabilidade de um sistema de controle pode ser analisada pelo estudo das raízes do polinômio característico do denominador da função de transferência em malha fechada. Estas raízes, também conhecidas como pólos do sistema, determinam como o sistema responde a diferentes entradas ao longo do tempo. A localização desses pólos no plano complexo indica se as respostas do sistema serão estáveis, instáveis ou em um estado de margem de estabilidade.

No caso do sistema considerado, a função de transferência em malha fechada resultante é calculada pelo produto e pela soma das funções de transferência do controlador proporcional, da planta e do sensor, estruturados como segue:
\[
    C(s)/R(s) = \frac{Gc \times Gp}{1 + Gc \times Gp \times Hs}
\]
A partir desta expressão, o denominador, que contém o polinômio característico do sistema, é dado por:
\[
    1 + Gc \times Gp \times Hs
\]
Expandindo e simplificando este produto usando os parâmetros e as funções definidas no código Scilab, o denominador da função de transferência em malha fechada é obtido. Este processo resulta em um polinômio em \(s\) que representa a combinação das dinâmicas do controlador, da planta e do sensor.

Para este sistema específico, utilizando os parâmetros dados, o denominador extraído e simplificado é:
\[
    0.5 + 0.92s + 1.3s^2 + s^3
\]
Este polinômio é então usado para a construção da matriz de Routh-Hurwitz, que é uma técnica clássica de análise de estabilidade, ajudando a determinar a presença de raízes com partes reais não negativas e, consequentemente, se o sistema é estável ou não.



\subsubsection{Cálculo da Matriz de Routh-Hurwitz}

Para construir a matriz de Routh-Hurwitz, utilizamos a função \texttt{routh\_t} que calcula essa matriz a partir do polinômio característico. A matriz de Routh-Hurwitz para o denominador do sistema é calculada e apresentada como segue:
\[
    \begin{array}{c|cc}
        s^3 & 1         & 0.92 \\
        s^2 & 1.3       & 0.5  \\
        s^1 & 0.5353846 & 0    \\
        s^0 & 0.5       &      \\
    \end{array}
\]

\subsubsection{Interpretação da Matriz de Routh-Hurwitz}

Os elementos da primeira coluna da matriz de Routh-Hurwitz indicam a estabilidade do sistema. Todos os elementos devem ser positivos para garantir estabilidade. A matriz mostra que todos os termos são positivos, sugerindo que o sistema é estável. Esta análise detalhada fornece confiança adicional na robustez do sistema sob a configuração de controle atual.

\subsubsection{Conclusão da Análise de Estabilidade}

A análise com a matriz de Routh-Hurwitz confirma que o sistema é estável sob as condições atuais. A positividade de todos os termos na primeira coluna da matriz assegura que não há raízes com partes reais positivas, o que implica em uma resposta do sistema estável e controlada. Essa conclusão é vital para garantir que o sistema opere de forma segura e eficaz, mantendo o desempenho desejado.