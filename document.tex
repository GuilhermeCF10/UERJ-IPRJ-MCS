% ===============================================================
% Document Class ================================================
\documentclass{article}

% ===============================================================
% Graphic Packages ==============================================
\usepackage{graphicx}       % Permite a inclusão de imagens no documento.
\usepackage{tikz}           % Ferramenta poderosa para criar gráficos programaticamente dentro do LaTeX.
\usetikzlibrary{calc}       % Extensão da biblioteca TikZ que permite cálculos mais complexos de coordenadas.

% ===============================================================
% Mathematical Tools ============================================
\usepackage{amsmath}        % Melhora a aparência e a flexibilidade de comandos matemáticos.
\usepackage{siunitx}        % Facilita o uso de unidades do Sistema Internacional e ajuda a formatar números complexos.

% ===============================================================
% Font and Text Appearance ======================================
\usepackage{mathptmx}       % Altera a fonte padrão do documento para Times New Roman.

% ===============================================================
% Table of Contents Customization ===============================
\usepackage{tocloft}  % Oferece controle total sobre a aparência das listas de conteúdos, figuras, tabelas, etc.

% ===============================================================
% Figure Positioning ============================================
\usepackage{float}     % Melhora a interface para definir o posicionamento de objetos flutuantes como figuras e tabelas.

% ===============================================================
% Paragraph Spacing and Indentation =============================
\usepackage{setspace}  % Permite o ajuste fino do espaçamento entre linhas.
\usepackage{indentfirst} % Adiciona indentação ao primeiro parágrafo de cada seção.

% ===============================================================
% Page Layout ===================================================
\usepackage[a4paper, top=3cm, bottom=2cm, left=3cm, right=2cm]{geometry}  % Define as margens de todo o documento.
\setlength{\parindent}{4em}  % Define o tamanho da indentação para todos os parágrafos.
\setlength{\emergencystretch}{3em}

% ========================================a=======================
% Section Heading Customization =================================
\makeatletter
\renewcommand\paragraph{\@startsection{paragraph}{4}{\z@}%
    {2ex plus 1ex minus .2ex}%
    {1em}%
    {\normalfont\normalsize\bfseries}}
\makeatother

% Código
% \usepackage{listings}
% \usepackage{xcolor}  % Permite a customização de cores
% \usepackage[utf8]{inputenc}

% Configuração básica para exibir código
% \lstset{ 
%   style=mystyle, 
%   framexleftmargin=3.5mm, 
%   frame=shadowbox, 
%   rulesepcolor=\color{black}, 
%   linewidth=0.6\linewidth, 
%   xleftmargin=12pt, 
%   aboveskip=12pt, 
%   belowskip=12pt 
% }
% \usepackage{listings}
% \usepackage{xcolor}  % Permite a customização de cores
% \usepackage{courier} % Fonte de estilo monoespaçado
% \usepackage{fontspec} % Somente necessário para XeLaTeX ou LuaLaTeX
% % \usepackage[utf8]{inputenc}

% \lstset{
%   backgroundcolor=\color[RGB]{249,246,239},   % Cor de fundo escolhida (exemplo: bege claro)
%   basicstyle=\ttfamily\footnotesize,  % Estilo de fonte básico
%   breaklines=true,  % Quebra automática de linha
%   frame=lines,  % Linhas na parte superior e inferior do código
%   numbers=left,  % Números de linha à esquerda
%   numberstyle=\tiny\color{gray},  % Estilo dos números de linha
%   keywordstyle=\color[RGB]{40,40,255},  % Cor dos comandos
%   commentstyle=\color[RGB]{0,125,0},  % Cor dos comentários
%   stringstyle=\color[RGB]{255,0,0},  % Cor das strings
%   morecomment=[l]{//}, % Definindo "//" como início de comentário de linha
%   showstringspaces=false,  % Não mostra espaços em strings como caracteres especiais
%   rulecolor=\color{black},  % Cor da moldura
%   captionpos=b,  % Posição da legenda (bottom)
%   abovecaptionskip=5pt,  % Espaço acima da legenda
%   belowcaptionskip=5pt,  % Espaço abaixo da legenda
%   xleftmargin=0.2\textwidth,  % Margem esquerda
%   xrightmargin=0.2\textwidth  % Margem direita
% }
\usepackage{listings}
\usepackage{xcolor}
\usepackage{courier}

\lstset{
  backgroundcolor=\color[RGB]{249,246,239},
  basicstyle=\ttfamily\footnotesize,
  breaklines=true,
  frame=single,
  numbers=left,
  numberstyle=\tiny\color{gray},
  keywordstyle=\color[RGB]{40,40,255},
  commentstyle=\color[RGB]{0,125,0},
  stringstyle=\color[RGB]{255,0,0},
  showstringspaces=false,
  rulecolor=\color{black},
  captionpos=b,
  abovecaptionskip=5pt,
  belowcaptionskip=5pt,
  xleftmargin=0.15\textwidth,
  xrightmargin=0.15\textwidth,
  morecomment=[l]{//}
}

% -- Carregando o pacote do bloco de código: 
% \usepackage{listings}
% \usepackage[utf8]{inputenc}
% % -- Formatação básica 
% \usepackage[utf8]{inputenc} 
% \usepackage[english]{babel} 
% \usepackage{times} 
% \setlength{\parindent}{8pt} 
% \usepackage{indentfirst}
% % -- Definindo cores: 
% \usepackage[dvipsnames]{xcolor} 
% \definecolor{codegreen}{rgb}{0,0.6,0} 
% \definecolor{codegray}{rgb}{0.5,0.5,0.5} 
% \definecolor{codepurple}{rgb}{0.58,0,0.82} 
% \definecolor{backcolour}{rgb}{0.95,0.95,0.92}
% % -- Definindo um estilo personalizado: 
% \lstdefinestyle{mystyle}{ 
%     backgroundcolor=\color{backcolour},    
%     commentstyle=\color{codepurple}, 
%     keywordstyle=\color{NavyBlue}, 
%     numberstyle=\tiny\color{codegray}, 
%     stringstyle=\color {codepurple}, 
%     basicstyle=\ttfamily\footnotesize\bfseries, 
%     breakatwhitespace=false,          
%     breaklines=true,                  
%     captionpos=t,                     
%     keepspaces=true,                  
%     numbers=left,                     
%     numbersep=5pt,                   
%     showspaces=false,                 
%     showstringspaces=false, 
%     showtabs=false,                   
%     tabsize =2
% }
% % -- Configurando o estilo personalizado: 
% \lstset{style=mystyle}



% ===============================================================
% Begin Document ================================================
\begin{document}

% ===============================================================
% Capa ==========================================================
% \begin{titlepage}
    \centering
    % Desenhar a margem
    \begin{tikzpicture}[remember picture, overlay]
        \draw[line width = 4pt] ($(current page.north west) + (20mm, -20mm)$) rectangle ($(current page.south east) + (-10mm,10mm)$);
        \draw[line width = 1pt] ($(current page.north west) + (21.5mm,-21.5mm)$) rectangle ($(current page.south east) + (-11.5mm,11.5mm)$);
    \end{tikzpicture}

    % Linha de logos
    \includegraphics[height=0.151\textwidth]{header/contra-capa/assets/uerj.png}\hfill
    \includegraphics[height=0.15\textwidth]{header/contra-capa/assets/iprj.jpeg}\hfill

    \vspace{2cm} % Espaço vertical após os logos

    % Informações do curso
    {\Large\bfseries Instituto Politécnico do Estado do Rio de Janeiro \par}
    \vspace{0.5cm}
    {\large Curso de Engenharia da Computação \par}

    \vspace{3cm} % Espaço vertical antes do título da atividade

    % Nome do estudante
    {\large\bfseries Guilherme Cagide Fialho \par}

    \vspace{3cm}

    % Título da atividade
    {\large\bfseries Projeto de Modelagem e Controle de Sistemas em Scilab \par}

    \vfill % Empurra o restante para o fundo da página

    % Local e ano
    {\large\bfseries Nova Friburgo \par}
    \vspace{0.3cm}
    {\large\bfseries 2024 \par}
\end{titlepage}
\newpage % Começa uma nova página após a capa


% ===============================================================
% Contra Capa ===================================================
% \begin{titlepage}
    \centering
    % Desenhar a margem
    \begin{tikzpicture}[remember picture, overlay]
        \draw[line width = 4pt] ($(current page.north west) + (20mm, -20mm)$) rectangle ($(current page.south east) + (-10mm,10mm)$);
        \draw[line width = 1pt] ($(current page.north west) + (21.5mm,-21.5mm)$) rectangle ($(current page.south east) + (-11.5mm,11.5mm)$);
    \end{tikzpicture}

    % Linha de logos
    \includegraphics[height=0.151\textwidth]{header/contra-capa/assets/uerj.png}\hfill
    \includegraphics[height=0.15\textwidth]{header/contra-capa/assets/iprj.jpeg}\hfill

    \vspace{2cm} % Espaço vertical após os logos

    % Informações do curso
    {\Large\bfseries Instituto Politécnico do Estado do Rio de Janeiro \par}
    \vspace{0.5cm}
    {\large Graduação em Engenharia da Computação \par}

    \vspace{3cm} % Espaço vertical antes do título da atividade

    % Nome do estudante
    {\large Guilherme Cagide Fialho \par}

    \vspace{1.5cm}

    % Título da atividade
    {\large\bfseries Projeto de Modelagem e Controle de Sistemas em Scilab \par}

    \vspace{1cm} % Espaço vertical antes do título da atividade

    % Informações do projeto
    \begin{flushright}
        \begin{minipage}{0.5\textwidth}
            \large
            \raggedleft % Garante que o texto dentro da minipage seja alinhado à direita
            Projeto de Conclusão da Disciplina Modelagem e Controle de Sistemas
        \end{minipage}
    \end{flushright}

    \vspace{1.5cm}

    % Informações do orientador
    \begin{flushleft}
        \begin{minipage}{0.5\textwidth}
            \large
            \raggedright
            Orientador: Prof. Joel Sánchez Domínguez
        \end{minipage}
    \end{flushleft}


    \vfill % Empurra o restante para o fundo da página

    % Local e ano
    {\large Nova Friburgo \par}
    \vspace{0.3cm}
    {\large 2024 \par}
\end{titlepage}
\newpage % Começa uma nova página após a capa


% ===============================================================
% Resumo ========================================================
% \begin{titlepage}
    \thispagestyle{empty} % Remove números de página
    \setstretch{1.5} % Espaçamento entre linhas, certifique-se de que o pacote setspace está incluído em document.tex

    \begin{center}
        \textbf{\Large RESUMO}
    \end{center}


    \vspace{1cm} % Espaço vertical

    \noindent CAGIDE FIALHO, G. Relatório do projeto de Modelagem
    e controle de sistemas. 2024. 54 f. Trabalho de Conclusão de Disciplina Modelagem e Controle de Sistemas (Graduação em
    Engenharia da computação) – Graduação em Engenharia da Computação, Universidade
    do Estado do Rio de Janeiro, Nova Friburgo, 2024.

    \vspace{0.4cm} % Espaço vertical

    Este trabalho explora a modelagem e o controle de um sistema dinâmico do tipo massa-mola-amortecedor, utilizando a plataforma Scilab para desenvolvimento e simulação. O foco do estudo está na implementação de modelos matemáticos para descrever a dinâmica do sistema e na análise de sua resposta sob diversas condições iniciais, sem a aplicação de forças externas. Utilizando a ferramenta Xcos, um componente gráfico do Scilab, realizamos simulações que permitiram uma análise visual e quantitativa das respostas transientes do sistema. O estudo destaca a influência dos parâmetros físicos, como a massa, o coeficiente de amortecimento, e a constante da mola, nas características de resposta do sistema. Além disso, técnicas de controle foram empregadas para ajustar a resposta do sistema, demonstrando como o amortecimento pode contribuir para a estabilização após perturbações e enfatizando a relevância de uma parametrização cuidadosa para alcançar um comportamento eficaz do sistema. Este projeto contribui para a compreensão das teorias de controle aplicáveis em sistemas mecânicos e outros contextos de sistemas dinâmicos na engenharia.
    \vspace{0.4cm} % Espaço vertical

    \textbf{Palavras-chave}: Modelagem e Controle, Sistema Massa-Mola-Amortecedor, Simulação, Scilab, Xcos.
\end{titlepage}


% ===============================================================
% Abstract ======================================================
% \begin{titlepage}
    \thispagestyle{empty} % Remove page numbers
    \setstretch{1.5} % Line spacing, make sure the setspace package is included in document.tex

    \begin{center}
        \textbf{\Large ABSTRACT}
    \end{center}

    \vspace{1cm} % Vertical space

    \noindent CAGIDE FIALHO, G. Report on the Modeling and Control Systems Project. 2024. \pageref{LastPage} p. Final Project for the Modeling and Control Systems Course (Bachelor’s degree in Computer Engineering) - Bachelor’s Degree in Computer Engineering, State University of Rio de Janeiro, Nova Friburgo, 2024.

    \vspace{0.4cm} % Vertical space

    This work explores the modeling and control of a dynamic system of the mass-spring-damper type, using the Scilab platform for development and simulation. The focus of the study is on implementing mathematical models to describe the system dynamics and analyzing its response under various initial conditions without the application of external forces. Using Xcos, a graphical component of Scilab, we conducted simulations that allowed for visual and quantitative analysis of the system's transient responses. The study highlights the influence of physical parameters, such as mass, damping coefficient, and spring constant, on the system's response characteristics. Furthermore, control techniques were employed to adjust the system's response, demonstrating how damping can contribute to stabilization after disturbances and emphasizing the importance of careful parameterization to achieve effective system behavior. This project contributes to the understanding of control theories applicable to mechanical systems and other dynamic system contexts in engineering.
    \vspace{0.4cm} % Vertical space

    \textbf{Keywords}: Modeling and Control, Mass-Spring-Damper System, Simulation, Scilab, Xcos.
\end{titlepage}


% ===============================================================
% Atividades ====================================================
% \section{Atividade 1}
\subsection{Descrição do Modelo}
O sistema modelado é um oscilador massa-mola-amortecedor, onde a massa está sujeita à força restauradora de uma mola e ao amortecimento proporcional à velocidade. A equação diferencial que descreve o movimento do sistema é dada por:
\[
    m \ddot{x} + C \dot{x} + Kx = 0
\]
onde \( x \) representa o deslocamento da massa \( m \) da sua posição de equilíbrio, \( \dot{x} \) é a velocidade, \( \ddot{x} \) é a aceleração, \( C \) é o coeficiente de amortecimento, e \( K \) é a constante da mola. A força de entrada é considerada nula, indicando que não há forças externas atuando sobre o sistema após o instante inicial.

\subsection{Parâmetros do Sistema}
Os parâmetros utilizados no modelo do sistema são especificados como segue:
\begin{itemize}
    \item Massa (\( m \)): 10 kg
    \item Coeficiente de amortecimento (\( C \)): 7 Ns/m
    \item Constante da mola (\( K \)): 5 N/m
\end{itemize}

\subsection{Condições Iniciais para a Simulação}
As condições iniciais para a simulação são detalhadas na tabela a seguir, baseadas nos parâmetros especificados acima:
\begin{center}
    \begin{tabular}{|c|c|c|}
        \hline
        \textbf{Caso} & \textbf{Velocidade Inicial \( V_0 \)} & \textbf{Posição Inicial \( X_0 \)} \\
        \hline
        1             & \( 5 \, \text{m/s} \)                 & \( 0 \, \text{m} \)                \\
        2             & \( 0 \, \text{m/s} \)                 & \( 2.5 \, \text{m} \)              \\
        3             & \( 3.33 \, \text{m/s} \)              & \( 2 \, \text{m} \)                \\
        \hline
    \end{tabular}
\end{center}

Esta tabela reflete os valores numéricos para cada caso, facilitando a compreensão e a aplicação direta dos parâmetros na simulação.

\subsection{Código Scilab para simular a resposta do sistema massa-mola-amortecedor}
\begin{lstlisting}[language=Scilab, caption=Código Scilab para simular a resposta do sistema massa-mola-amortecedor]
    // Definicao das principais variaveis do sistema fisico
    m = 10;  // massa
    c = 7;   // coeficiente de amortecimento
    k = 5;   // constante da mola

    // Funcao que define o sistema de equacoes diferenciais (EDO) para o modelo massa-mola-amortecedor
    function dxdt = sistema(t, x)
    // x(1) representa o deslocamento, x(2) representa a velocidade
    // Esta funcao retorna a derivada da velocidade e do deslocamento, respectivamente
    dxdt = [x(2); -c/m * x(2) - k/m * x(1)];
    endfunction

    // Configuracao do intervalo de tempo para a simulacao
    t0 = 0; // Tempo inicial (s)
    tf = 20; // Tempo final (s)
    t = linspace(t0, tf, 1000); // Cria um vetor de tempo linearmente espacado para a simulacao

    // Definicao das condicoes iniciais para cada caso de simulacao
    condicoes_iniciais = [
    m/5, m/3; // Caso 3: posicao inicial (m) e velocidade inicial (m/s)
    m/4, 0;   // Caso 2: posicao inicial (m) e velocidade inicial (m/s)
    0, m/2;   // Caso 1: posicao inicial (m) e velocidade inicial (m/s)
    ];

    // Cores designadas para cada caso de simulacao para facilitar a visualizacao
    cores = ['#007bff', '#dc3545', '#8B4513']; // Azul, vermelho, marrom

    // Loop para executar e plotar cada caso de simulacao separadamente
    for i = 1:3
        x0 = condicoes_iniciais(i, :)'; // Transpoe as condicoes iniciais para a formatacao correta
        sol = ode(x0, t0, t, sistema); // Resolve a EDO usando o metodo de ODE

        scf(i); // Cria uma nova figura para cada iteracao
        plot(t, sol(1, :),  'color', cores(i),  'LineWidth', 2); // Plot do deslocamento x(t)
        xlabel('Tempo (s)'); // Etiqueta do eixo X
        ylabel('Deslocamento x(t)'); // Etiqueta do eixo Y
        title(['Resposta do Sistema para o Caso ', string(i)]); // Titulo do grafico
        legend('x(t)', "location", "best"); // Legenda
        xgrid(); // Ativa a grade no grafico
    end

    // Preparacao do grafico combinado
    scf(); // Cria uma nova figura
    clf(); // Limpa a figura atual
    xlabel('Tempo (s)');
    ylabel('Deslocamento x(t)');
    title('Resposta do Sistema para Todos os Casos');
    xgrid(); // Ativando a grade

    // Execucao da simulacao para cada caso e plotagem no mesmo grafico
    for i = 1:3
    x0 = condicoes_iniciais(i, :)'; // Condicoes iniciais para o caso i (transposto para coluna)
    sol = ode(x0, t0, t, sistema); // Resolvendo a equacao diferencial

    // Plotando os resultados com cores definidas
    plot(t, sol(1, :), 'color', cores(i), 'LineWidth', 2);
    end

    // Criar a legenda detalhando cada caso
    legend(['Caso 1: x0 = 0, v0 = m/2', 'Caso 2: x0 = m/4, v0 = 0', 'Caso 3: x0 = m/5, v0 = m/3'], "location", "best");
\end{lstlisting}

\subsection{Análise dos Resultados}
Cada um dos casos de simulação foi configurado com condições iniciais distintas para explorar como o sistema responde a diferentes estados iniciais de deslocamento e velocidade.

\subsubsection{Caso 1: Velocidade Inicial Elevada}
\begin{figure}[H]
    \centering
    \includegraphics[width=0.6\textwidth]{atividades/1-atividade/assets/caso1.png}
    \caption{Resposta do sistema para o Caso 1}
\end{figure}
No Caso 1, o sistema é inicialmente impulsionado com uma alta velocidade (\(5 \, \text{m/s}\)), partindo do repouso (\(X_0 = 0\)). Esta condição inicial leva a uma resposta inicialmente enérgica, onde a massa oscila com uma amplitude elevada, seguida de um rápido decaimento energético devido ao amortecimento significativo (\(C = 7 \, \text{Ns/m}\)). O amortecimento não só reduz a amplitude das oscilações rapidamente, mas também garante que o sistema não persista em um estado de oscilação prolongada, estabilizando-se em um tempo curto.

\subsubsection{Caso 2: Deslocamento Inicial Sem Velocidade}
\begin{figure}[H]
    \centering
    \includegraphics[width=0.6\textwidth]{atividades/1-atividade/assets/caso2.png}
    \caption{Resposta do sistema para o Caso 2}
\end{figure}
O Caso 2 é caracterizado por um deslocamento inicial (\(2.5 \, \text{m}\)) sem impulso inicial de velocidade (\(V_0 = 0\)). Aqui, observamos uma resposta típica de um sistema oscilatório subamortecido onde o sistema retorna ao equilíbrio através de oscilações que decaem gradativamente. Este caso destaca como a energia potencial armazenada na mola é convertida em energia cinética e dissipada pelo amortecedor. As oscilações decrescem em amplitude mais gradualmente do que no Caso 1, demonstrando uma transferência de energia mais prolongada antes da estabilização.

\subsubsection{Caso 3: Velocidade e Deslocamento Iniciais}
\begin{figure}[H]
    \centering
    \includegraphics[width=0.6\textwidth]{atividades/1-atividade/assets/caso3.png}
    \caption{Resposta do sistema para o Caso 3}
\end{figure}
No Caso 3, o sistema inicia com condições iniciais moderadas tanto de velocidade (\(3.33 \, \text{m/s}\)) quanto de deslocamento (\(2 \, \text{m}\)). Esta configuração produz uma resposta dinâmica complexa, onde a interação entre energia cinética e potencial é mais evidente. A amplitude inicial é significativa, com uma taxa de decaimento que ilustra eficientemente o papel do amortecimento. As oscilações observadas são mais sustentadas que no Caso 1, mas menos intensas do que no Caso 2, refletindo um equilíbrio entre as energias cinética e potencial no início da simulação.

\subsubsection{Comparação Unificada dos Casos}
\begin{figure}[H]
    \centering
    \includegraphics[width=0.6\textwidth]{atividades/1-atividade/assets/caso-all-in-one.png}
    \caption{Resposta unificada do sistema para os Casos 1, 2 e 3}
\end{figure}
A análise unificada dos três casos demonstra de forma clara as diferenças significativas nas respostas do sistema decorrentes de diversas condições iniciais. A seguir, discutiremos detalhadamente cada resposta e suas implicações para a compreensão do comportamento dinâmico do sistema:

\begin{itemize}
    \item \textbf{Caso 1 (Azul Escuro)}: Iniciado com uma alta velocidade inicial (\(5 \, \text{m/s}\)) e sem deslocamento inicial, este caso exibe a maior amplitude de oscilação observada. A energia cinética inicial é rapidamente convertida em energia potencial pela mola, resultando em oscilações de grande amplitude que são rapidamente amortecidas. Este caso ilustra o efeito de um forte amortecimento, onde a energia é dissipada rapidamente, levando a um retorno rápido à posição de equilíbrio sem oscilações residuais prolongadas. Esta configuração é ideal em situações onde a rápida estabilização após distúrbios é crucial, como em sistemas de suspensão de veículos.

    \item \textbf{Caso 2 (Vermelho)}: Com um deslocamento inicial (\(2.5 \, \text{m}\)) e sem velocidade inicial, o sistema mostra uma resposta clássica de um oscilador subamortecido. A energia potencial armazenada na mola é convertida gradualmente em energia cinética, com a energia sendo dissipada ao longo do tempo pelo amortecedor. As oscilações decaem suavemente, refletindo uma conversão mais lenta de energia que é típica em aplicações onde é necessário manter uma certa quantidade de movimento ou onde oscilações graduais são preferíveis, como em alguns tipos de sensores mecânicos.

    \item \textbf{Caso 3 (Marrom)}: Este caso combina condições iniciais moderadas de velocidade (\(3.33 \, \text{m/s}\)) e deslocamento (\(2 \, \text{m}\)), resultando numa resposta dinâmica mais complexa que engloba características dos dois primeiros casos. A amplitude inicial é significativa, mas as oscilações são mais controladas e decaem de maneira gradual. Este caso destaca a importância do equilíbrio entre rigidez da mola e amortecimento no projeto de sistemas mecânicos, onde é necessário um compromisso entre estabilidade rápida e manutenção de energia dinâmica.
\end{itemize}

Esta comparação detalhada destaca não apenas a influência das condições iniciais na resposta do sistema, mas também o papel crítico do amortecimento e da rigidez da mola na determinação da natureza da resposta dinâmica. A análise fornece insights valiosos para o design e a otimização de sistemas mecânicos em engenharia, sublinhando a necessidade de uma seleção cuidadosa de parâmetros de acordo com os requisitos específicos de cada aplicação.


\subsection{Comentários Gerais e Conclusão}
Os gráficos e análises ilustram claramente como as condições iniciais impactam a resposta dinâmica do sistema massa-mola-amortecedor. A energia inicial, seja como deslocamento ou velocidade, define a resposta imediata do sistema, mostrando a complexidade do comportamento de sistemas dinâmicos lineares. Observamos que o amortecimento é essencial para reduzir as oscilações e trazer o sistema de volta ao repouso de maneira eficiente, sublinhando sua importância no design de componentes mecânicos.

A adequação do coeficiente de amortecimento e da rigidez da mola é crucial para otimizar sistemas para suas funções específicas, como a absorção de choques em suspensões de veículos ou a precisão em instrumentos de medição. Além disso, a análise das condições iniciais é vital no planejamento e teste de sistemas mecânicos, onde engenheiros e designers devem antecipar cenários variados de operação.

Este estudo destaca a necessidade de um entendimento profundo das dinâmicas de sistemas para inovação em engenharia, proporcionando uma base sólida para a compreensão dos princípios de mecânica e dinâmica que são fundamentais no design de sistemas controlados e mecanismos em geral.

% \section{Atividade 2: Simulação com Xcos}
\subsection{Descrição do Modelo e Ferramentas}
Nesta atividade, utilizamos o Xcos, uma ferramenta gráfica do Scilab para a simulação de sistemas dinâmicos. O Xcos permite a construção de diagramas de blocos que facilitam a visualização e implementação do sistema massa-mola-amortecedor com diferentes entradas e condições iniciais.

\subsection{Parâmetros do Sistema}
O sistema é descrito pelos seguintes parâmetros, que são consistentes com os usados na Atividade 1:
\begin{itemize}
    \item Massa (\( m \)): 10 kg
    \item Coeficiente de amortecimento (\( C \)): 7 Ns/m
    \item Constante da mola (\( K \)): 5 N/m
\end{itemize}

\subsection{Condições Iniciais de Simulação}
As simulações foram executadas sob várias condições iniciais para explorar a resposta do sistema sob diferentes estados iniciais. A seguir estão as condições iniciais utilizadas, incluindo uma condição inicial adicional específica para esta atividade (Caso 0):

\begin{center}
    \begin{tabular}{|c|c|c|}
        \hline
        \textbf{Caso} & \textbf{Velocidade Inicial \( V_0 \)} & \textbf{Posição Inicial \( X_0 \)} \\
        \hline
        0             & \( 0 \, \text{m/s} \)                 & \( 0 \, \text{m} \)                \\
        1             & \( 5 \, \text{m/s} \)                 & \( 0 \, \text{m} \)                \\
        2             & \( 0 \, \text{m/s} \)                 & \( 2.5 \, \text{m} \)              \\
        3             & \( 3.33 \, \text{m/s} \)              & \( 2 \, \text{m} \)                \\
        \hline
    \end{tabular}
\end{center}

Esta tabela facilita a referência rápida às condições iniciais para cada caso simulado, permitindo uma comparação mais direta entre os diferentes cenários testados.

\subsection{Diagrama de Blocos no Xcos}
\begin{figure}[H]
    \centering
    \includegraphics[width=0.8\textwidth]{atividades/2-atividade/assets/diagrama.png}
    \caption{Diagrama de blocos utilizado na simulação no Xcos.}
\end{figure}

\subsection{Resultados e Análise}


% Caso 0 =========================================================
\subsubsection{Análise dos Resultados para o Caso 0}
No Caso 0, analisamos a resposta do sistema quando ele parte de condições completamente estáticas (\(V_0 = 0 \, \text{m/s}\) e \(X_0 = 0 \, \text{m}\)). Esta configuração é vital para avaliar a resposta pura do sistema a uma entrada controlada sem influência inicial de deslocamento ou velocidade.

\paragraph{Deslocamento}
\begin{figure}[H]
    \centering
    \includegraphics[height=0.7\textwidth]{atividades/2-atividade/assets/deslocamento-caso-0.png}
    \caption{Gráfico de deslocamento para o Caso 0.}
\end{figure}
O gráfico de deslocamento revela um pico máximo de aproximadamente 4.7 unidades aos 5.1 segundos, marcando o tempo de pico. O tempo de subida, definido como o intervalo para atingir o primeiro pico máximo a partir do repouso, é, portanto, cerca de 5.1 segundos. Após atingir o pico, o sistema exibe oscilações amortecidas que rapidamente reduzem em amplitude. O tempo de estabelecimento, onde as oscilações ficam dentro de uma faixa de ±2\% do valor final, é aproximadamente de 18 segundos, após o qual o sistema entra em uma zona estacionária, indicando estabilidade.

\paragraph{Velocidade}

\begin{figure}[H]
    \centering
    \includegraphics[height=0.7\textwidth]{atividades/2-atividade/assets/velocidade-caso-0.png}
    \caption{Gráfico de velocidade para o Caso 0.}
\end{figure}
O gráfico de velocidade reflete a resposta imediata do sistema à força aplicada. A velocidade atinge um pico negativo de cerca de -1.54 unidades em torno de 6.2 segundos, o que corresponde ao tempo de pico para a velocidade. A velocidade oscila abaixo e acima de zero, indicando a resposta oscilatória do sistema ao deslocamento. As oscilações diminuem progressivamente e o sistema alcança a zona estacionária por volta de 18 segundos, estabilizando-se completamente em zero.

\paragraph{Comentários Gerais}
A análise do Caso 0 mostra como o sistema responde a um estímulo externo na ausência de condições iniciais de energia. Os parâmetros transitórios, como tempo de subida, pico, e de estabelecimento, juntamente com a observação da zona estacionária, são cruciais para entender a dinâmica do sistema e a eficácia do amortecimento em trazer o sistema de volta ao repouso, minimizando oscilações excessivas. Este caso estabelece uma base comparativa para outros casos com condições iniciais variadas.


% Caso 1 =========================================================
\subsubsection{Análise dos Resultados para o Caso 1}
No Caso 1, analisamos a resposta do sistema quando ele parte com uma velocidade inicial significativa (\(V_0 = 5 \, \text{m/s}\)) e sem deslocamento inicial (\(X_0 = 0 \, \text{m}\)). Esta condição inicial permite avaliar como uma energia cinética inicial afeta a resposta dinâmica do sistema, especialmente em termos de deslocamento máximo e oscilações resultantes.


\paragraph{Deslocamento}
\begin{figure}[H]
    \centering
    \includegraphics[height=0.7\textwidth]{atividades/2-atividade/assets/deslocamento-caso-1.png}
    \caption{Gráfico de deslocamento para o Caso 1.}
\end{figure}
O gráfico mostra que o sistema parte de zero e rapidamente atinge um pico de aproximadamente 6.5 unidades ao redor de 2.7 segundos, refletindo uma resposta aguda à velocidade inicial. Esse pico é seguido por uma diminuição significativa, que desce abaixo do zero antes de estabilizar. O tempo de subida é rapidamente alcançado, enquanto o tempo de estabelecimento, onde as oscilações permanecem dentro de uma faixa de ±2\% do valor estacionário final, é observado por volta de 18 segundos.


\paragraph{Velocidade}
\begin{figure}[H]
    \centering
    \includegraphics[height=0.7\textwidth]{atividades/2-atividade/assets/velocidade-caso-1.png}
    \caption{Gráfico de velocidade para o Caso 1.}
\end{figure}
A velocidade inicialmente picos a uma taxa significativa, refletindo o impulso inicial aplicado. O pico máximo de velocidade ocorre quase simultaneamente com o pico de deslocamento, marcando -1.54 unidades em torno de 1.68 segundos. Após atingir este pico, a velocidade oscila e gradualmente se aproxima de zero, indicando que o sistema está alcançando uma zona estacionária por volta de 18 segundos, semelhante ao observado no deslocamento.


\paragraph{Comentários Gerais}
A análise do Caso 1 ilustra como a condição inicial de velocidade influencia a resposta dinâmica do sistema massa-mola-amortecedor. Os parâmetros transitórios, como o tempo de subida e o tempo de pico, são drasticamente diferentes em comparação com o Caso 0, onde não havia energia cinética inicial. Isso destaca a importância de considerar condições iniciais variadas para entender completamente o comportamento do sistema em diferentes cenários de operação. Este caso também reforça o papel crítico do amortecimento na estabilização do sistema após perturbações iniciais.


% Caso 2 =========================================================
\subsubsection{Análise dos Resultados para o Caso 2}
No Caso 2, analisamos a resposta do sistema quando ele parte com um deslocamento inicial (\(X_0 = 2.5 \, \text{m}\)) e sem velocidade inicial (\(V_0 = 0 \, \text{m/s}\)). Esta configuração é fundamental para entender como o sistema responde a uma perturbação inicial na posição sem impulso inicial.

\paragraph{Deslocamento}
\begin{figure}[H]
    \centering
    \includegraphics[height=0.7\textwidth]{atividades/2-atividade/assets/deslocamento-caso-2.png}
    \caption{Gráfico de deslocamento para o Caso 2.}
\end{figure}
O gráfico de deslocamento mostra que o sistema parte de um deslocamento inicial de 2.5 m, rapidamente atinge um pico de cerca de 4.25 m aos 5.2 segundos, indicando a resposta máxima do sistema ao ser liberado. Após esse pico, o sistema exibe oscilações que rapidamente se amortecem, com o deslocamento oscilando abaixo e acima do zero, estabilizando-se finalmente em torno do zero. O tempo de estabelecimento, onde as oscilações permanecem dentro de uma faixa de ±2\% do valor final, é aproximadamente de 18 segundos.

\paragraph{Velocidade}
\begin{figure}[H]
    \centering
    \includegraphics[height=0.7\textwidth]{atividades/2-atividade/assets/velocidade-caso-2.png}
    \caption{Gráfico de velocidade para o Caso 2.}
\end{figure}
A velocidade inicialmente aumenta à medida que o sistema se move de volta para a posição de equilíbrio, atingindo um pico negativo de -0.52 m/s logo após o início, correspondente à velocidade máxima ao passar pelo equilíbrio na direção oposta ao deslocamento inicial. A velocidade então oscila, diminuindo em magnitude devido ao amortecimento, até estabilizar-se em zero. O sistema atinge uma zona estacionária com velocidade quase nula, demonstrando a eficácia do amortecimento em dissipar a energia cinética inicialmente induzida pelo deslocamento.

\paragraph{Comentários Gerais}
O Caso 2 destaca a resposta do sistema a um teste de posição, com deslocamento inicial sem velocidade inicial. Os resultados mostram claramente como a energia potencial armazenada é convertida em energia cinética, e como o amortecimento é crucial para a estabilização do sistema. Este caso também é importante para verificar a eficácia do sistema em retornar ao repouso sem oscilações residuais prolongadas, essencial em aplicações práticas onde respostas rápidas e estabilizadas são necessárias.

% Caso 3 =========================================================
\subsubsection{Análise dos Resultados para o Caso 3}
No Caso 3, analisamos a resposta do sistema quando ele parte com uma velocidade inicial (\(V_0 = 3.33 \, \text{m/s}\)) e um deslocamento inicial (\(X_0 = 2 \, \text{m}\)). Esta combinação de condições iniciais é significativa para explorar a resposta dinâmica sob energia cinética e potencial simultâneas.

\paragraph{Deslocamento}
\begin{figure}[H]
    \centering
    \includegraphics[height=0.7\textwidth]{atividades/2-atividade/assets/deslocamento-caso-3.png}
    \caption{Gráfico de deslocamento para o Caso 3.}
\end{figure}
O gráfico de deslocamento mostra que o sistema começa com um impulso inicial que o leva a um pico de aproximadamente 5.75 m ao redor de 2.4 segundos. Após esse pico, o sistema exibe oscilações que reduzem gradualmente em amplitude devido ao amortecimento. O sistema estabiliza perto do zero, com o tempo de estabelecimento aproximadamente em 18 segundos, onde as oscilações ficam dentro de uma faixa aceitável indicando uma zona estacionária.

\paragraph{Velocidade}
\begin{figure}[H]
    \centering
    \includegraphics[height=0.7\textwidth]{atividades/2-atividade/assets/velocidade-caso-3.png}
    \caption{Gráfico de velocidade para o Caso 3.}
\end{figure}
A velocidade inicialmente mostra uma rápida ascensão, atingindo um pico de aproximadamente 5.32 m/s. Essa alta velocidade inicial contribui para o rápido pico de deslocamento observado. A velocidade então oscila, diminuindo progressivamente até estabilizar-se em torno de zero. A estabilização final da velocidade é alcançada em torno de 18 segundos, refletindo a eficácia do amortecimento e a interação entre as forças restauradoras e o amortecimento.

\paragraph{Comentários Gerais}
O Caso 3 oferece uma perspectiva complexa sobre a dinâmica do sistema quando energias cinética e potencial são ambas significativas desde o início. As oscilações observadas e a subsequente estabilização demonstram como diferentes tipos de energia inicial influenciam a resposta do sistema e a eficácia do amortecimento em controlar a resposta até a estabilidade. Este caso é particularmente útil para entender a resposta do sistema em condições iniciais variadas e complexas, sendo essencial para aplicações práticas onde o sistema pode ser sujeito a perturbações iniciais múltiplas.

% Conclusão
\subsection{Conclusão Geral dos Casos Estudados}

Ao longo desta atividade, analisamos as respostas do sistema massa-mola-amortecedor sob várias condições iniciais, abrangendo os Casos 0 a 3. Cada caso foi projetado para ilustrar aspectos diferentes da dinâmica do sistema, considerando diferentes combinações de deslocamento e velocidade iniciais.

\paragraph{Observações Gerais}

Os casos estudados mostraram uma ampla gama de comportamentos dinâmicos:
\begin{itemize}
    \item \textbf{Caso 0} serviu como um ponto de referência, onde o sistema partiu do repouso sem energia inicial, permitindo observar a resposta pura à força aplicada.
    \item \textbf{Caso 1} demonstrou a influência de uma velocidade inicial significativa, ilustrando como a energia cinética influencia as oscilações e a estabilidade subsequente do sistema.
    \item \textbf{Caso 2} focou no efeito de um deslocamento inicial sem velocidade, enfatizando a conversão de energia potencial em energia cinética e vice-versa.
    \item \textbf{Caso 3} combinou tanto deslocamento quanto velocidade iniciais, mostrando a interação complexa entre as duas formas de energia desde o início da simulação.
\end{itemize}

Durante as simulações, os parâmetros do sistema foram mantidos constantes para garantir a consistência dos resultados, permitindo uma comparação direta entre os diferentes casos. Os resultados foram meticulosamente analisados para observar o comportamento transiente e a estabilidade a longo prazo, utilizando métricas como tempo de subida, tempo de pico e tempo de estabelecimento. As oscilações foram avaliadas para determinar a eficácia do amortecimento em dissipar a energia e estabilizar o sistema.

\paragraph{Conclusões da Análise}
Esta atividade sublinhou a importância de compreender a dinâmica de sistemas massa-mola-amortecedor em várias configurações iniciais. As simulações forneceram insights valiosos sobre como diferentes condições iniciais afetam a resposta do sistema e como o design adequado do amortecimento e da rigidez da mola é crucial para o comportamento desejado. A abordagem utilizada garantiu que todas as premissas da atividade fossem cumpridas, fornecendo uma base sólida para futuras investigações e aplicações práticas dos princípios estudados.

% \section{Atividade 3}

\subsection{Descrição do Modelo e Análise de Sistema}
Nesta atividade, desenvolvemos e analisamos a função de transferência de um sistema massa-mola-amortecedor, utilizando os seguintes parâmetros específicos, essenciais para entender a dinâmica do sistema:
\begin{itemize}
    \item Massa (\( m \)): 10 kg, que influi diretamente na inércia do sistema, afetando como o sistema responde a forças externas.
    \item Coeficiente de amortecimento (\( C \)): 7 Ns/m, crucial para atenuar as oscilações e determinar a rapidez com que o sistema atinge um estado de equilíbrio.
    \item Constante da mola (\( K \)): 5 N/m, que define a rigidez do sistema e afeta a frequência das oscilações naturais.
\end{itemize}
A função de transferência modelada é expressa por:
\[
    G(s) = \frac{1}{10s^2 + 7s + 5}
\]

\subsection{Cálculo dos Polos e Parâmetros do Sistema}
Os polos da função de transferência são essenciais para entender como o sistema responde a estímulos externos:
\begin{itemize}
    \item Polo 1: \( -0.35 + 0.614j \)
    \item Polo 2: \( -0.35 - 0.614j \)
\end{itemize}
Estes polos indicam uma resposta oscilatória amortecida, característica de um sistema subamortecido devido à sua parte real negativa e parte imaginária não nula.

Os parâmetros do sistema de segunda ordem são determinados como segue:
\begin{itemize}
    \item Frequência natural não-amortecida (\( \omega_n \)): 0.707 rad/s, que descreve a frequência natural de oscilação do sistema na ausência de amortecimento.
    \item Coeficiente de amortecimento (\( \zeta \)): 0.495, refletindo a eficácia do amortecimento em reduzir as oscilações.
    \item Ganho estático (\( K_p \)): 0.2, representando a resposta do sistema em estado estacionário a uma entrada de degrau unitário.
\end{itemize}

\subsection{Resposta ao Impulso}
Utilizando o software Scilab, simulamos a resposta ao impulso do sistema, como ilustrado abaixo. A resposta apresenta um pico inicial significativo seguido por um decaimento exponencial das oscilações, um comportamento típico de sistemas subamortecidos.
\begin{figure}[H]
    \centering
    \includegraphics[width=0.8\textwidth]{atividades/3-atividade/assets/resposta-ao-impulso.png}
    \caption{Resposta ao impulso do sistema massa-mola-amortecedor}
\end{figure}

\subsection{Discussão}
A análise dos polos e dos parâmetros do sistema demonstra que ele é bem projetado para equilibrar uma resposta rápida com oscilações controladas, minimizando as oscilações excessivas sem comprometer a agilidade da resposta. Esta característica é crucial para sistemas de controle que exigem precisão e estabilidade.

\subsection{Conclusões}
Esta atividade ofereceu uma visão profunda sobre como os parâmetros físicos — massa, amortecimento e rigidez — influenciam a resposta dinâmica de um sistema. Estes insights são fundamentais para o design e a análise de sistemas de controle adequados, que são essenciais em aplicações práticas onde a precisão e estabilidade são críticas.

% \section{Atividade 4}

\subsection{Descrição do Modelo e Simulação}
Nesta atividade, analisamos um sistema de controle típico conforme descrito no diagrama de blocos abaixo. O controlador utilizado é um controlador proporcional com ganho \( K = \frac{m}{3} \). A função de transferência da planta corresponde à função de transferência da Atividade 3. O sensor é modelado por um sistema de primeira ordem com ganho \( K_s = 1 \) e constante de tempo \( T_s = \frac{m}{6} \).

\subsection{Construção do Diagrama de Blocos}
O diagrama de blocos do sistema é mostrado abaixo:
\begin{figure}[H]
    \centering
    % Inserir diagrama de blocos aqui
    % \includegraphics[width=0.7\textwidth]{path/to/diagrama_blocos.png}
    \caption{Diagrama de blocos do sistema de controle}
    \label{fig:diagrama_blocos}
\end{figure}

As funções de transferência são definidas como:
\begin{itemize}
    \item \( G_c(s) = \frac{m}{3} = \frac{10}{3} \)
    \item \( G_p(s) = \frac{1}{10 s^2 + 7 s + 5} \)
    \item \( H(s) = \frac{1}{1 + \frac{10}{6} s} = \frac{1}{1 + 1.6667 s} \)
\end{itemize}

\subsection{Função de Transferência em Malha Fechada}
A função de transferência em malha fechada \( C(s)/R(s) \) é obtida pela relação:
\[
    G_{closed}(s) = \frac{G_c(s) G_p(s)}{1 + G_c(s) G_p(s) H(s)}
\]
Substituindo as expressões das funções de transferência, temos:
\[
    G_{closed}(s) = \frac{\frac{10}{3} \cdot \frac{1}{10 s^2 + 7 s + 5}}{1 + \frac{10}{3} \cdot \frac{1}{10 s^2 + 7 s + 5} \cdot \frac{1}{1 + 1.6667 s}}
\]

Simplificando a expressão:
\[
    G_{closed}(s) = \frac{0.6 + s}{0.9 + 0.92s + 1.3s^2 + s^3}
\]

\subsection{Análise de Estabilidade pelo Critério de Routh-Hurwitz}
A análise de estabilidade é realizada utilizando o critério de Routh-Hurwitz. Os coeficientes do polinômio do denominador da função de transferência em malha fechada foram extraídos e utilizados na matriz de Routh-Hurwitz para determinar a estabilidade do sistema.

\subsubsection{Resultados da Análise de Estabilidade}
A matriz de Routh-Hurwitz calculada é:
\[
    RH\_matrix = \begin{bmatrix}
        0.9  & ... \\
        0.92 & ... \\
        1.3  & ... \\
        1    & ... \\
    \end{bmatrix}
\]

O sistema é estável, pois todos os elementos da primeira coluna da matriz de Routh-Hurwitz são positivos.

\subsection{Análise de Estabilidade para Diferentes Valores de \( K \)}
Para diferentes valores do ganho do controlador \( K \), a estabilidade do sistema foi verificada. A análise mostrou que o sistema é estável para todos os valores de \( K_p \) testados (de 1 a 10).

\subsubsection{Resultados da Resposta ao Degrau}
Abaixo, mostramos o gráfico da resposta ao degrau para diferentes valores de \( K_p \):

\begin{figure}[H]
    \centering
    \includegraphics[width=0.7\textwidth]{4-atividade/assets/impulsos-diferentes-kp.png}
    \caption{Resposta ao degrau para diferentes valores de \( K_p \)}
    \label{fig:resposta-degrau-kp}
\end{figure}

\subsection{Conclusões}
A análise da resposta ao degrau e da estabilidade utilizando a matriz de Routh-Hurwitz mostrou que o sistema massa-mola-amortecedor controlado proporcionalmente é estável para os valores de \( K_p \) entre 1 e 10. O comportamento da resposta ao degrau varia significativamente com o ganho do controlador, como esperado, demonstrando a importância de ajustar corretamente o ganho do controlador para atingir a resposta desejada do sistema.

% \input{atividades/4-atividade/latex/new-atividade4.tex}

\section{Atividade 4}

Esta atividade consiste na modelagem e análise de um sistema de controle massa-mola-amortecedor. O sistema é controlado por um controlador proporcional e monitorado por um sensor de primeira ordem. O objetivo é analisar a estabilidade do sistema e determinar o limite crítico do sistema.


% ===============================================================
% Letra A =======================================================
\subsection{Parte (a): Descrição do Diagrama de Blocos}

\begin{figure}[H]
    \centering
    \includegraphics[width=0.8\textwidth]{atividades/4-atividade/assets/diagrama-blocos.png}
    \caption{Diagrama de blocos do sistema de controle proporcional.}
    \label{fig:diagrama_blocos}
\end{figure}

O diagrama de blocos apresentado na Figura~\ref{fig:diagrama_blocos} ilustra a configuração do sistema de controle:

\begin{itemize}
    \item \textbf{Controlador Proporcional (\(G_c(s)\))}: Com ganho de \(\frac{10}{3}\) = (\(3.333\)), o controlador ajusta a saída com base na diferença entre a referência e o sinal medido pelo sensor.
    \item \textbf{Planta (\(G_p(s)\))}: Representada pela função de transferência \(\frac{1}{10s^2 + 7s + 5}\), que descreve a dinâmica do sistema massa-mola-amortecedor.
    \item \textbf{Sensor (\(H(s)\))}: O sensor é modelado como um sistema de primeira ordem com a função de transferência \(\frac{1}{1 + 1.6667s}\), capturando a resposta da variável controlada com uma certa constante de tempo.
    \item \textbf{Soma (\(\Sigma\))}: Um somador que computa a diferença entre a referência e a saída do sensor, alimentando essa diferença para o controlador.
    \item \textbf{Realimentação}: O loop de realimentação é crucial para garantir que a saída do sistema esteja em conformidade com a entrada desejada.
\end{itemize}

O sistema é projetado para monitorar e ajustar a saída de modo a atingir um estado desejado, com foco na estabilidade e eficiência do controle.



% ===============================================================
% Letra B =======================================================
\subsection{Parte (b): Função de Transferência em Malha Fechada}

Para o sistema de controle proposto, primeiramente definimos os parâmetros físicos e as configurações do sistema. A planta é um sistema massa-mola-amortecedor, e o controlador utilizado é um controlador proporcional. O sensor é modelado por um sistema de primeira ordem. Os parâmetros são definidos como segue:

\begin{itemize}
    \item Massa, \(m = 10\, \text{kg}\)
    \item Coeficiente de amortecimento, \(c = 7\, \text{Ns/m}\)
    \item Constante da mola, \(k = 5\, \text{N/m}\)
\end{itemize}

O ganho do controlador proporcional, \(K\), é definido como:
\[ K = \frac{m}{3} \]

A constante de tempo do sensor, \(T_s\), é determinada por:
\[ T_s = \frac{m}{6} \]

As funções de transferência para a planta (\(G_p(s)\)), o sensor (\(H_s(s)\)), e o controlador proporcional (\(G_c(s)\)) são definidas da seguinte forma:
\begin{align*}
    G_p(s) & = \frac{1}{m s^2 + c s + k} = \frac{1}{10 s^2 + 7 s + 5} \\
    H_s(s) & = \frac{1}{T_s s + 1} = \frac{1}{\frac{10}{6} s + 1}     \\
    G_c(s) & = K = \frac{10}{3}
\end{align*}

\subsubsection{Código Scilab utilizado calcular a Função de Transferência em Malha Fechada}
\begin{lstlisting}[language=Scilab, caption=Código Scilab para calcular a função de transferência em malha fechada]
    // Definicao dos parametros
    s = poly(0, 's');
    m = 10;  // massa
    c = 7;   // coeficiente de amortecimento
    k = 5;   // constante da mola
    
    // Definicao das funcoes de transferencia
    K = m / 3;  // Ganho do controlador proporcional
    Ts = m / 6;  // Constante de tempo do sensor
    
    // Funcoes de Transferencia
    Gp = syslin('c', 1, m*s^2 + c*s + k);  // Planta
    Hs = syslin('c', 1, Ts*s + 1);  // Sensor
    Gc = syslin('c', K, 1);  // Controlador Proporcional
    
    // Funcao de Transferencia em Malha Fechada C(s)/R(s)
    sys = Gc * Gp / (1 + Gc * Gp * Hs);
    
    // Exibicao da Funcao de Transferencia em Malha Fechada
    disp("Funcao de Transferencia em Malha Fechada C(s)/R(s):");
    disp(sys);
    \end{lstlisting}



A função de transferência em malha fechada \(C(s)/R(s)\) é calculada pela integração dessas funções de transferência, resultando na seguinte expressão:
\[
    C(s)/R(s) = \frac{G_c(s) G_p(s)}{1 + G_c(s) G_p(s) H_s(s)}
\]

Após a simplificação e cálculos realizados pelo software Scilab, a função de transferência em malha fechada resultante é:
\[
    \frac{0.2 + 0.3333333s}{0.5 + 0.92s + 1.3s^2 + s^3}
\]

Esta função de transferência em malha fechada indica como o sistema responde à entrada \(R(s)\) dada a configuração de controle atual. A expressão mostra a relação entrada-saída considerando o feedback do sensor e a ação do controlador proporcional.


% ===============================================================
% Letra C =======================================================

\subsection{Parte (c): Análise de Estabilidade com o Critério de Routh-Hurwitz}

Após calcular a função de transferência em malha fechada \(C(s)/R(s)\), o próximo passo é analisar a estabilidade do sistema de controle. Utilizamos o critério de Routh-Hurwitz para essa finalidade, que é uma técnica fundamental na teoria de controle para determinar a estabilidade de um sistema linear.


\subsubsection{Código Scilab para a Análise de Routh-Hurwitz}
\begin{lstlisting}[language=Scilab, caption=Código Scilab para calcular a matriz de Routh-Hurwitz]
    // Definicao dos parametros
    s = poly(0, 's');
    m = 10;  // massa
    c = 7;   // coeficiente de amortecimento
    k = 5;   // constante da mola

    // Definicao das funcoes de transferencia
    K = m / 3;  // Ganho do controlador proporcional
    Ts = m / 6;  // Constante de tempo do sensor

    // Funcoes de Transferencia
    Gp = syslin('c', 1, m*s^2 + c*s + k);  // Planta
    Hs = syslin('c', 1, Ts*s + 1);  // Sensor
    Gc = syslin('c', K, 1);  // Controlador Proporcional

    // Funcao de Transferencia em Malha Fechada C(s)/R(s)
    sys = Gc * Gp / (1 + Gc * Gp * Hs);

    // Extraindo o denominador da Funcao de Transferencia para Analise de Estabilidade
    den = sys.den;

    rh_matrix = routh_t(den);

    // Exibir a Matriz de Routh-Hurwitz
    disp("Matriz de Routh-Hurwitz:");
    disp(rh_matrix);
    \end{lstlisting}


\subsubsection{Extração do Denominador da Função de Transferência}

A estabilidade do sistema pode ser analisada através do denominador da função de transferência em malha fechada, dado que as raízes do polinômio característico no denominador determinam a resposta do sistema. O denominador extraído é:
\[ 0.62 + 1s + 1.1s^2 + s^3 \]


\subsubsection{Cálculo da Matriz de Routh-Hurwitz}

Para construir a matriz de Routh-Hurwitz, utilizamos a função \texttt{routh\_t} que calcula essa matriz a partir do polinômio característico. A matriz de Routh-Hurwitz para o denominador do sistema é calculada e apresentada como segue:
\[
    \begin{array}{c|cc}
        s^3 & 1          & 0.92 \\
        s^2 & 1.3        & 0.5  \\
        s^1 & 0.5353846  & 0    \\
        s^0 & 0.5        &      \\
    \end{array}
\]

\subsubsection{Interpretação da Matriz de Routh-Hurwitz}

Os elementos da primeira coluna da matriz de Routh-Hurwitz indicam a estabilidade do sistema. Todos os elementos devem ser positivos para garantir estabilidade. A matriz mostra que todos os termos são positivos, sugerindo que o sistema é estável. Esta análise detalhada fornece confiança adicional na robustez do sistema sob a configuração de controle atual.

\subsubsection{Conclusão da Análise de Estabilidade}

A análise com a matriz de Routh-Hurwitz confirma que o sistema é estável sob as condições atuais. A positividade de todos os termos na primeira coluna da matriz assegura que não há raízes com partes reais positivas, o que implica em uma resposta do sistema estável e controlada. Essa conclusão é vital para garantir que o sistema opere de forma segura e eficaz, mantendo o desempenho desejado.

% ===============================================================
% Letra D =======================================================
\subsection{Parte (d): Análise de Estabilidade para Diferentes Valores de \(K\)}

A estabilidade do sistema de controle é investigada para uma variação do ganho \(K\) do controlador proporcional, substituído pelo parâmetro variável \(K\). Utilizamos a função de transferência em malha fechada definida pelos parâmetros físicos do sistema para determinar para quais valores de \(K\) o sistema é estável.

\subsubsection{Definição da Função de Transferência}
Com base nos parâmetros do sistema, a função de transferência da planta \(G_p(s)\) e do sensor \(H_s(s)\) são definidas como segue:

\[
    G_p(s) = \frac{1}{10 s^2 + 7 s + 5}
\]

\[
    H_s(s) = \frac{1}{\frac{10}{6} s + 1}
\]

\subsubsection{Função de Transferência em Malha Fechada}
A função de transferência em malha fechada \(T(s)\), considerando o controlador proporcional \(G_c(s) = K\), é dada por:

\[
    T(s) = \frac{K \cdot G_p(s)}{1 + K \cdot G_p(s) \cdot H_s(s)}
\]

Substituindo \(G_c(s)\), \(G_p(s)\), e \(H_s(s)\) com os valores acima, obtemos:

\[
    T(s) = \frac{K \left(\frac{1}{10 s^2 + 7 s + 5}\right)}{1 + K \left(\frac{1}{10 s^2 + 7 s + 5}\right) \left(\frac{1}{\frac{10}{6}s + 1}\right)}
\]

Multiplicando numerador e denominador pelo MMC dos denominadores das funções de transferência, obtemos:

\[
    T(s) = \frac{5Ks + 3K}{3K + 50s^3 + 65s^2 + 46s + 15}
\]
Esta função representa a resposta do sistema em função do ganho proporcional \(K\), onde \(K\) modula a entrada em função das dinâmicas combinadas da planta e do sensor.

\subsubsection{Construção da Matriz de Routh-Hurwitz}

A análise da estabilidade do sistema é feita através da matriz de Routh-Hurwitz, que é construída a partir do polinômio característico:
\[
    50s^3 + 65s^2 + 46s + 15 + 3K
\]

A estabilidade do sistema é analisada através da construção da matriz de Routh-Hurwitz para o polinômio característico derivado do denominador da função de transferência em malha fechada:
\[
    \begin{array}{c|cc}
        s^3 & 50                     & 46      \\
        s^2 & 65                     & 15 + 3K \\
        s^1 & \frac{150K - 2240}{65} & 0       \\
        s^0 & 15 + 3K                &         \\
    \end{array}
\]
Onde:
\[
    s^1 = \frac{150K - 2240}{65} = 2.3077K - 34.4615
\]

\subsubsection{Análise de Condições de Estabilidade}
Para garantir a estabilidade, todos os coeficientes na primeira coluna da matriz de Routh-Hurwitz devem ser positivos:
\begin{itemize}
    \item \(s^3 = 50\) é constantemente positivo.
    \item \(s^2 = 65\) é positivo.
    \item \(s^1 = 2.3077K - 34.4615 > 0\), o que requer que \(K\) seja menor que \(\frac{34.4615}{2.3077} \approx 14.93\) para manter a positividade deste termo. Assim, a estabilidade é assegurada para \(K < 14.93\).
    \item \(s^0 = 15 + 3K > 0\), que é trivialmente satisfeito desde que \(K > -5\), mas a condição mais restritiva vem de \(s^1\).
\end{itemize}

\subsubsection{Conclusão da Análise de Condições de Estabilidade}
A análise meticulosa da matriz de Routh-Hurwitz indica que o sistema mantém a estabilidade quando o ganho proporcional, \(K\), está dentro do intervalo especificado. Valores de \(K\) superiores a 14.93 podem induzir instabilidade, manifestando-se através de oscilações não amortecidas ou respostas exageradas a perturbações, comprometendo tanto a performance quanto a segurança operacional do sistema.

Assim, é fundamental que \(K\) seja cuidadosamente escolhido para manter-se dentro do intervalo \(0 < K < 14.93\) para assegurar um comportamento estável e previsível do sistema em todas as condições operacionais.

% \section{Atividade 5}

\subsection{Descrição do Modelo e Simulação}
Nesta atividade, simulamos um sistema de controle que envolve um sistema massa-mola-amortecedor com um controlador proporcional. O sistema é descrito pela seguinte equação diferencial:
\[
    m\frac{d^2x(t)}{dt^2} + c\frac{dx(t)}{dt} + kx(t) = f(t),
\]
onde \( m = 10 \), \( c = 7 \), e \( k = 5 \).

\subsection{Construção do Diagrama de Blocos}
O diagrama de blocos para o sistema é apresentado a seguir, ilustrando como os componentes do sistema — controlador, planta e sensor — estão interligados.

\begin{figure}[H]
    \centering
    \includegraphics[width=0.7\textwidth]{5-atividade/assets/diagrama-a.png}
    \caption{Diagrama de blocos do sistema de controle para Atividade 5}
    \label{fig:diagrama_blocos_5}
\end{figure}

\subsection{Simulação do Sistema}
Para a simulação, utilizamos um sinal de degrau com amplitude \( A = \frac{m}{4} = 2.5 \). O tempo de simulação foi definido em 50 segundos para permitir a observação completa da resposta do sistema.

\subsubsection{Configuração da Simulação}
O sinal de degrau foi configurado para iniciar em 0 e atingir 2.5 no instante \( t = 1 \) segundo. O tempo de simulação total foi estabelecido para 50 segundos para assegurar que a resposta do sistema fosse completamente observada.

\begin{figure}[H]
    \centering
    \includegraphics[width=0.8\textwidth]{5-atividade/assets/diagrama-b.png}
    \caption{Diagrama de blocos utilizado para a simulação}
    \label{fig:diagrama_blocos_b}
\end{figure}
\subsubsection{Resultados da Simulação}
A resposta do sistema ao degrau é apresentada na figura abaixo, onde são destacados o tempo de subida, tempo de pico, tempo de acomodação e a zona estacionária, utilizando a ferramenta DataTip para marcar esses pontos significativos.

\begin{figure}[H]
    \centering
    \includegraphics[height=0.7\textwidth]{5-atividade/assets/simulation-b.png}
    \caption{Resposta do sistema ao degrau com configuração de amplitude \( A = 2.5 \)}
    \label{fig:simulation_5b}
\end{figure}

\subsection{Análise Detalhada dos Resultados da Simulação}
A resposta do sistema ao degrau é apresentada na Figura \ref{fig:simulation_5b}, demonstrando as dinâmicas chave do sistema controlado. Analisamos detalhadamente cada parte da resposta:

\begin{itemize}
    \item \textbf{Tempo de Subida:} O tempo de subida refere-se ao intervalo necessário para que a resposta do sistema suba do estado inicial até um determinado percentual do valor final, geralmente 90\%. No gráfico, o sistema leva aproximadamente 4.8 segundos para atingir um valor próximo de 1.55, que é o primeiro pico significativo. Este comportamento inicial mostra como o sistema responde rapidamente ao degrau, com a energia inicialmente absorvida e depois liberada pela combinação de massa, mola e amortecedor.
    \item \textbf{Tempo de Pico:} O pico ocorre no momento em que a saída atinge seu valor máximo em resposta ao degrau. O primeiro pico de 1.55 é atingido em torno de 5 segundos após a aplicação do degrau, ilustrando a máxima extensão da resposta do sistema antes de começar a amortecer devido às forças de fricção e à força restauradora da mola.

    \item \textbf{Tempo de Acomodação:} Após o pico inicial, o sistema começa a se estabilizar, reduzindo as oscilações até alcançar um estado quase constante. Este período é crucial, pois mostra a eficácia do amortecimento em dissipar a energia inicialmente induzida. No gráfico, o sistema mostra sinais de acomodação em torno de 28 segundos, indicando que o amortecimento e a rigidez da mola estão bem dimensionados para controlar as oscilações.

    \item \textbf{Zona Estacionária:} O sistema é considerado em estado estacionário quando as oscilações em torno do valor de equilíbrio se tornam negligíveis. No gráfico, isso é observado após aproximadamente 28 segundos, onde a saída mantém-se constante em cerca de 0.998. Esta fase é fundamental para avaliar se o sistema atingiu o equilíbrio desejado após a perturbação inicial.
\end{itemize}


\textbf{Conclusões da Análise:} A resposta ao degrau revela que o sistema massa-mola-amortecedor, equipado com um controlador proporcional, consegue retornar a um estado de equilíbrio após uma perturbação inicial. A análise destaca a importância de um ajuste apropriado do amortecimento e da rigidez da mola para assegurar que o sistema não apenas retorne ao equilíbrio, mas que o faça de maneira eficiente e sem oscilações excessivas. Este comportamento é indicativo de um sistema bem projetado, capaz de manter a estabilidade mesmo sob condições iniciais desafiadoras.


\subsection{Simulação com Diferentes Configurações de Ganho e Amplitude}
Nesta seção, expandimos a simulação para avaliar o impacto de diferentes configurações de ganho do controlador e amplitude do sinal de entrada. Três casos distintos foram simulados:


\begin{enumerate}
    \item \textbf{Caso Base (Amplitude A=1, Ganho=3.333)}: Mostrado pela linha verde no gráfico.
    \item \textbf{Caso com A=2.5 e Ganho=3.333}: Mostrado pela linha amarela no gráfico.
    \item \textbf{Caso com A=2.5 e Ganho=6.666}: Mostrado pela linha azul no gráfico.
\end{enumerate}

\begin{figure}[H]
    \centering
    \includegraphics[width=0.8\textwidth]{5-atividade/assets/diagrama-c.png}
    \caption{Diagrama de blocos utilizado para a simulação dos três casos}
    \label{fig:diagrama_blocos_c}
\end{figure}

\subsubsection{Análise dos Resultados}
Os resultados das simulações são visualizados no gráfico seguinte, onde diferentes cores representam os diferentes casos testados.



\begin{figure}[H]
    \centering
    \includegraphics[width=0.8\textwidth]{5-atividade/assets/simulation-c.png}
    \caption{Resposta do sistema para diferentes configurações de ganho e amplitude}
    \label{fig:response_comparison}
\end{figure}

\begin{itemize}
    \item \textbf{Verde (Caso Base)}: A resposta é bastante atenuada, com um pico máximo de aproximadamente 0.311 e acomodação rápida. Este caso mostra a capacidade do sistema de controlar eficazmente pequena perturbações.

    \item \textbf{Amarelo (A=2.5, Ganho=3.333)}: Com o aumento da amplitude, o sistema apresenta um overshoot maior, atingindo aproximadamente 1.554, com oscilações mais pronunciadas antes de estabilizar perto de 1.002. Isso indica que a resposta é mais vigorosa devido à maior entrada, mas ainda gerenciável.

    \item \textbf{Azul (A=2.5, Ganho=6.666)}: O aumento do ganho resulta em um overshoot significativamente maior, cerca de 2.683, com oscilações prolongadas que se estendem ao longo de todo o período de simulação. A resposta é mais agressiva e menos estável, demonstrando que um ganho mais alto pode introduzir instabilidade no sistema.
\end{itemize}

\subsection{Comparação e Comentários sobre as Respostas}
A comparação entre os três casos ilustra claramente a influência da amplitude do sinal de entrada e do ganho do controlador sobre a dinâmica do sistema. As principais observações são:

\begin{itemize}
    \item \textbf{Impacto do Aumento da Amplitude}: O aumento da amplitude do degrau de 1 para 2.5 resulta em um maior overshoot e tempo de acomodação mais longo, o que é esperado em sistemas de controle devido à maior energia introduzida no sistema.

    \item \textbf{Efeitos do Aumento do Ganho do Controlador}: Ao dobrar o ganho do controlador de 3.333 para 6.666, enquanto mantendo a amplitude elevada, observa-se uma resposta muito mais volátil e um pico de overshoot quase dobrado. Isso sugere que embora um ganho mais alto possa ser benéfico para uma resposta mais rápida, também pode comprometer a estabilidade geral do sistema.

    \item \textbf{Conclusões}: Os resultados indicam que um ajuste cuidadoso do ganho é crucial, especialmente em sistemas onde a estabilidade é uma preocupação. Para aplicações que requerem respostas rápidas e podem tolerar algum overshoot, um ganho mais alto pode ser apropriado. No entanto, para a maioria das aplicações industriais e comerciais, um ganho mais moderado e uma abordagem balanceada são recomendados para evitar oscilações excessivas e garantir a estabilidade do sistema.
\end{itemize}

Conclusivamente, esta análise demonstra a importância de um design de controlador bem ponderado, ressaltando a necessidade de equilibrar resposta rápida e estabilidade, dependendo dos requisitos específicos da aplicação.

% \section{Atividade 6}
\subsection{Introdução ao Modelo com Controlador PID}
Os controladores PID são amplamente reconhecidos por sua eficácia e flexibilidade, combinando três elementos distintos para obter um desempenho superior: proporcional, integral e derivativo. Ao contrário dos controladores proporcionais, que ajustam a resposta do sistema de maneira direta ao erro atual, os controladores PID aproveitam três abordagens diferentes, cada uma desempenhando uma função específica.

O componente proporcional funciona de modo semelhante ao controlador proporcional simples, ajustando a saída do sistema em relação direta ao erro, com o objetivo de reduzir a diferença entre o valor medido e o valor desejado. No entanto, quando o componente proporcional sozinho não consegue corrigir totalmente o erro acumulado, entra em ação o componente integral, que soma e integra o erro ao longo do tempo para eliminá-lo.

Além disso, o componente derivativo desempenha um papel crucial ao prever mudanças no erro, ajudando a evitar que essas variações causem impactos negativos na saída do sistema. Com a integração desses três elementos, os controladores PID conseguem oferecer um controle mais preciso e estável, ajustando continuamente a saída para manter o sistema no estado desejado.
A fórmula padrão de um controlador PID pode ser representada pela equação \ref{eq:pid}:
\begin{equation}
u(t) = K_p e(t) + K_i \int_{0}^{t} e(T) dT + K_d \frac{d e(t)}{dt}
\label{eq:pid}
\end{equation}

O método de Ziegler-Nichols, desenvolvido por John G. Ziegler e Nathaniel B. Nichols, é uma técnica consolidada para a sintonia de controladores PID. Este método é particularmente útil porque simplifica a configuração dos controladores ao fornecer fórmulas práticas para calcular os ganhos \( K_p \), \( K_i \), e \( K_d \) com base na resposta do sistema a uma entrada de teste. Esses parâmetros são ajustados para otimizar a resposta do sistema em termos de tempo de subida, sobreposição e tempo de assentamento.

Os valores dos ganhos são estabelecidos de acordo com a estabilidade observada do sistema e são tipicamente calculados a partir do ganho crítico \( K_c \) e do período crítico \( P_c \), que são obtidos através de testes de malha aberta. A Tabela \ref{tab:ziegler-nichols} resume os valores recomendados para cada tipo de ganho:

\begin{table}[h]
\centering
\begin{tabular}{ccc}
\hline
\( K_p \) & \( K_i \) & \( K_d \) \\
\hline
\( 0,6 \times K_c \) & \( \frac{2 \times K_p}{P_c} \) & \( 0,125 \times P_c \) \\
\hline
\end{tabular}
\caption{Valores dos ganhos segundo o método de Ziegler-Nichols}
\label{tab:ziegler-nichols}
\end{table}

Esses parâmetros foram aplicados no ajuste do sistema, resultando na configuração que proporcionou a melhor resposta transitória sem causar instabilidade, como pode ser observado no gráfico da figura \ref{fig:instavel-posicao-sistema-controlador}. Este gráfico ilustra a resposta do sistema quando submetido a diferentes configurações de ganho. Foi constatado que um ganho crítico de \( K_c = 14.959 \) é o limite para a estabilidade do sistema. Valores superiores resultaram em uma resposta instável, indicando a importância de uma sintonia cuidadosa.

\begin{figure}[H]
    \centering
    \includegraphics[width=0.7\textwidth]{6-atividade/assets/instavel-posicao-sistema-controlador.png}
    \caption{Análise da instabilidade da posição do sistema com controlador PID ajustado para diferentes ganhos}
    \label{fig:instavel-posicao-sistema-controlador}
\end{figure}

\section{Determinação dos Parâmetros do Controlador PID}
Após identificar o ganho crítico \( K_c = 14.959 \) por meio de simulações, utilizamos o método de Ziegler-Nichols para definir os parâmetros do controlador PID. Para completar o ajuste dos parâmetros, é essencial determinar o período crítico \( P_c \), que é um indicador do comportamento oscilatório do sistema sem amortecimento.

\subsection{Cálculo dos Parâmetros do Controlador PID}
Utilizamos o método de Ziegler-Nichols, amplamente reconhecido por sua eficácia na sintonia inicial de controladores PID. Este método utiliza o ganho crítico \( K_c \) e o período crítico \( P_c \) para determinar os parâmetros de controle, ajustando a resposta do sistema em termos de estabilidade e rapidez.

\begin{itemize}
    \item O \textbf{ganho proporcional} \( K_p \) é calculado como:
    \[
    K_p = 0.6 \times K_c = 0.6 \times 14.959 = 8.9754
    \]

    \item O \textbf{ganho integral} \( K_i \) é calculado como:
    \[
    K_i = \frac{2 \times K_p}{P_c} = \frac{2 \times 8.9754}{P_c} = \frac{17.9508}{P_c}
    \]
    Assumindo um \( P_c \) conhecido, por exemplo, \( P_c = 10 \) (substitua este valor pelo seu valor real),
    \[
    K_i = \frac{17.9508}{10} = 1.79508
    \]

    \item O \textbf{ganho derivativo} \( K_d \) é calculado como:
    \[
    K_d = 0.125 \times P_c = 0.125 \times 10 = 1.25
    \]
\end{itemize}

\subsection{Implementação e Validação dos Parâmetros}
Os parâmetros calculados \( K_p = 8.9754 \), \( K_i = 1.79508 \), e \( K_d = 1.25 \) são implementados no controlador PID no ambiente de simulação, como o Scilab. Estes valores ajustam o sistema para responder adequadamente sob diversas condições operacionais, melhorando a estabilidade e a precisão.

A validação dos parâmetros através de simulações confirmará sua eficácia em manter o desempenho desejado do sistema, assegurando que o controle PID seja eficiente e eficaz.

\begin{figure}[H]
    \centering
    \includegraphics[width=0.8\textwidth]{6-atividade/assets/pid-implementado-com-valores-iniciais.png}
    \caption{Resposta do sistema com os parâmetros do PID ajustados, demonstrando a eficácia do controle em manter a estabilidade e a precisão do sistema.}
    \label{fig:resposta-pid}
\end{figure}

Subsequentemente, a simulação e a análise de resposta validam o desempenho do sistema sob o controle PID ajustado. As simulações ajudam a verificar se os parâmetros calculados são efetivos em manter a saída do sistema próxima ao valor desejado, sob diversas condições operacionais, garantindo a eficácia e a eficiência do controlador.

\subsection{Refinamento dos Parâmetros do Controlador PID}
Os parâmetros iniciais \( K_p \), \( K_i \), e \( K_d \) obtidos pelo método de Ziegler-Nichols, baseados no ganho crítico estimado de \( K_c = 14.959 \), fornecem um ponto de partida útil para a configuração do controlador PID. No entanto, a precisão inicial na estimativa de \( K_c \) pode influenciar diretamente a eficácia destes parâmetros, necessitando de ajustes refinados para alinhar o desempenho do controlador às características específicas do sistema.

\begin{itemize}
    \item \textbf{Ajuste do Ganho Proporcional (\( K_p \))}: O valor inicial de \( K_p = 8.9754 \) pode precisar ser ajustado se a resposta do sistema for muito lenta ou rápida demais, o que indica que a estimativa de \( K_c \) pode não ter capturado perfeitamente as dinâmicas do sistema.
    \item \textbf{Ajuste do Ganho Integral (\( K_i \))}: Da mesma forma, o valor de \( K_i = 1.79508 \) (calculado com um \( P_c \) hipotético de 10) pode requerer modificações para otimizar a correção de erros de longo prazo, sugerindo que a sensibilidade do sistema a erros acumulados pode ter sido subestimada.
    \item \textbf{Ajuste do Ganho Derivativo (\( K_d \))}: O valor inicial de \( K_d = 1.25 \) pode também necessitar de ajustes para melhor controlar a resposta do sistema a mudanças rápidas nas condições de entrada ou perturbações.
\end{itemize}

Recomenda-se a realização de testes operacionais adicionais para validar e ajustar esses parâmetros. O ajuste fino deve ser guiado por uma avaliação contínua da resposta do sistema, ajustando os parâmetros para atingir uma resposta ótima em termos de estabilidade, precisão e rapidez. Este processo iterativo é essencial para assegurar que o controlador PID atenda às exigências específicas do sistema e opere eficazmente em todas as condições previstas.

Este enfoque nos ajustes necessários reflete a compreensão de que, embora o método de Ziegler-Nichols ofereça uma excelente base teórica, a aplicação prática em sistemas dinâmicos reais muitas vezes requer uma personalização cuidadosa para alcançar os melhores resultados.

% \subsection{Descrição do Modelo com Controlador PID}
% Nesta atividade, modificamos o sistema de controle anterior para incluir um controlador PID. O sistema massa-mola-amortecedor com um controlador PID é descrito pela seguinte equação diferencial:
% \[
%     m\frac{d^2x(t)}{dt^2} + c\frac{dx(t)}{dt} + kx(t) = f(t) + K_p e(t) + K_i \int e(t) dt + K_d \frac{de(t)}{dt},
% \]
% onde \( m = 10 \), \( c = 7 \), \( k = 5 \), e os parâmetros PID são ajustados conforme necessário.

% \subsection{Diagrama base da questão 5}
% Na questão como diagrama base tinhamos o degrau base com valor final de 1, o que durante o decorrer da atividade foi solicitado o uso do degrau como Amplitude do degrau como A=m/4, como m=10, nesse caso a amplitude do degrau aacabou indo para A=2.5, vamos partir desse presuposto nesse caso aqui e agora, oonde nesse caso iremos abordar o diagrama, e iremos ali em busca da oscilação frequente para que possamos ter ali sua base de valor apropriado manual para obtermos os parametros do PID

% \begin{figure}[H]
%     \centering
%     \includegraphics[width=0.7\textwidth]{6-atividade/assets/diagrama-a.png}
%     \caption{Diagrama de blocos do sistema de controle com controlador PID}
%     \label{fig:diagrama_blocos_pid}
% \end{figure}

% \subsubsection{Adaptação do PID para detectar valor do ganho crítico proporcional do controlador}
% Nesse caso foi testado inúmeros valores no ganho através de diversas simulações e foi constatado que Nesse caso através de avaliação manual também, o valor de ganho crítico do controlador
% proporcinal foi 14,959, quaisquer valores maiores desestabilizaram o sistema, como visto
% no gráfico abaixo:


% \subsection{Construção do Diagrama de Blocos com Controlador PID}
% O diagrama de blocos para o sistema com o controlador PID é apresentado a seguir:

% \begin{figure}[H]
%     \centering
%     \includegraphics[width=0.7\textwidth]{6-atividade/assets/diagrama-a.png}
%     \caption{Diagrama de blocos do sistema de controle com controlador PID}
%     \label{fig:diagrama_blocos_pid}
% \end{figure}

% \subsection{Determinação dos Parâmetros do Controlador PID}
% Os parâmetros \(K_p\), \(K_i\) e \(K_d\) do controlador PID foram inicialmente ajustados usando uma combinação de abordagem empírica e as regras de Ziegler-Nichols. O processo foi o seguinte:

% 1. \textbf{Definição do Ganho Proporcional \(K_p\)}:
% Utilizamos o valor do ganho proporcional do controlador anterior como ponto de partida: \(K_p = 3.333\).

% 2. \textbf{Determinação dos Parâmetros Críticos}:
% Aumentamos o ganho proporcional até que o sistema apresentasse oscilações sustentadas. Supondo que o ganho crítico \(K_u\) fosse 5 e o período de oscilação crítica \(P_u\) fosse 10 segundos, utilizamos esses valores para calcular os parâmetros PID iniciais.

% 3. \textbf{Cálculo dos Parâmetros PID com as Regras de Ziegler-Nichols}:
% Com base nos valores \(K_u\) e \(P_u\), aplicamos as fórmulas de Ziegler-Nichols para determinar os parâmetros iniciais do controlador PID:
% \[
%     K_p = 0.6 \times K_u = 0.6 \times 5 = 3
% \]
% \[
%     K_i = \frac{2 \times K_p}{P_u} = \frac{2 \times 3}{10} = 0.6
% \]
% \[
%     K_d = 0.125 \times K_p \times P_u = 0.125 \times 3 \times 10 = 3.75
% \]

% 4. \textbf{Ajustes Finais dos Parâmetros}:
% Com base na resposta inicial do sistema, ajustamos os parâmetros \(K_i\) e \(K_d\) para melhorar a resposta. Os valores finais configurados foram:
% \begin{itemize}
%     \item Proporcional (\(K_p\)): 3.333
%     \item Integral (\(K_i\)): 0.6666
%     \item Derivativo (\(K_d\)): 4.16625
% \end{itemize}

% \subsection{Configuração da Simulação}
% Para a simulação, utilizamos um sinal de degrau com amplitude \( A = \frac{m}{4} = 2.5 \). O tempo de simulação foi definido em 50 segundos para permitir a observação completa da resposta do sistema. Os parâmetros PID foram inicialmente configurados como \( K_p = 3.333 \), \( K_i = 0.6666 \), e \( K_d = 4.16625 \).

% \subsection{Resultados da Simulação}
% A resposta do sistema ao degrau com o controlador PID é apresentada na figura abaixo, onde são destacados o tempo de subida, tempo de pico, tempo de estabilização e a zona estacionária, utilizando a ferramenta DataTip para marcar esses pontos significativos.

% % \begin{figure}[H]
% %     \centering
% %     \includegraphics[height=0.7\textwidth]{6-atividade/assets/simulation-pid.png}
% %     \caption{Resposta do sistema ao degrau com controlador PID}
% %     \label{fig:simulation_pid}
% % \end{figure}

% \subsection{Análise dos Resultados}
% A resposta do sistema ao degrau com o controlador PID é analisada, destacando como os parâmetros PID influenciam a estabilidade, o overshoot e o tempo de estabilização do sistema.

% \begin{itemize}
%     \item \textbf{Tempo de Subida:} O tempo de subida é o intervalo necessário para que a resposta do sistema suba do estado inicial até o primeiro pico significativo. Com os valores \( K_p = 3.333 \), \( K_i = 0.6666 \), e \( K_d = 4.16625 \), o sistema levou aproximadamente 5 segundos para atingir o primeiro pico, mostrando uma resposta rápida.

%     \item \textbf{Overshoot:} O gráfico mostra um overshoot inicial, onde a resposta ultrapassa o valor de referência. Este comportamento é típico quando os componentes derivativos e integrais são ajustados para melhorar a resposta dinâmica.

%     \item \textbf{Tempo de Estabilização:} O sistema leva cerca de 28-30 segundos para estabilizar próximo ao valor desejado, indicando a eficácia do controlador PID em atingir o valor de referência após a perturbação inicial.

%     \item \textbf{Zona Estacionária:} O sistema se aproxima de uma zona estacionária com pequenas oscilações, mostrando a necessidade de ajustes finos nos valores de \( K_i \) e \( K_d \) para eliminar qualquer erro estacionário residual e minimizar as oscilações.
% \end{itemize}

% \subsection{Ajustes Futuros}
% Com base na análise, se a resposta ainda precisar ser refinada, considere os seguintes ajustes:

% \begin{itemize}
%     \item \textbf{Reduzir \( K_d \) (Derivativo):} Se a oscilação persistir, reduza o valor de \( K_d \) para atenuar a resposta derivativa.
%     \item \textbf{Ajustar \( K_i \) (Integral):} Se houver um erro estacionário persistente, ajuste \( K_i \) para melhorar a eliminação do erro ao longo do tempo.
% \end{itemize}

% \subsection{Ajustes dos Valores}

% \begin{itemize}
%     \item Proporcional (Kp): 3.333 (mantido do controlador proporcional anterior)
%     \item Integral (Ki): 0.6666
%     \item Derivativo (Kd): 4.16625 (ajustado inicialmente)
% \end{itemize}

% \subsection{Simulação Futura}
% Após ajustes dos valores \( K_i \) e \( K_d \) conforme necessário, execute novamente a simulação e observe a resposta. Ajuste iterativamente até que o sistema atinja a resposta desejada.

% \subsection{Conclusão}
% A inclusão do controlador PID permite um ajuste mais fino da resposta do sistema, oferecendo uma maneira eficaz de melhorar a estabilidade e o desempenho do sistema de controle. Os ajustes de \( K_i \) e \( K_d \) são críticos para minimizar oscilações e alcançar a estabilidade desejada.

% Essa abordagem detalhada para determinar os parâmetros PID e ajustar a resposta do sistema pode ser documentada e usada como uma referência para ajustes futuros.


% ===============================================================
% End Document ==================================================
\end{document}
